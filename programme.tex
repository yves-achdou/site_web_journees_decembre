\documentclass[11pt]{article}
\usepackage{graphicx, pstricks, textpos, epsfig}
\usepackage{float}
% \title{Mean Field Games and Related Topics - 3\\
% Institut Henri Poincar{\'e}, Paris}
% \date{June 10-12 2015}
\begin{document}
\[
\includegraphics[width=\linewidth]{affiche}
\]
\vfill\eject
\section*{Schedule}

\subsection*{Wednesday June  10: Amphith{\'e}{\^a}tre Darboux, IHP}
\begin{itemize}
\item 9.00  Registration 
\item 9.20 - 9.30  Opening
\item 9.30 - 10.15  {\bf P.-L. Lions:}  TBA.
\item 10.15 - 10.50  {\bf R. Carmona:}  Models of Bank Runs as Mean Field Games of Timing. 
\item 10.50 - 11.20   coffee break
\item 11.20 - 11.55  {\bf R. Malham{\' e}:}   Cooperative and Non Cooperative Mean Field Game Methods in Energy Systems 
\item 11.55 - 12.30  {\bf F. Delarue:}  The master equation and the convergence problem in mean-field games.
\item 12.30 - 14.00  Lunch break
\item 14.00 -14.35 {\bf  M. Bardi:} Mean field games models of segregation.
\item 14.35 - 15.10 {\bf M. Fischer:} On the connection between mean field games and N-player games. 
\item 15.10 - 15.40  coffee break
\item 15.40 - 16.15  {\bf  A. Porretta:} On the weak theory for mean field games systems with local coupling. 
\item 16.15 - 16.50 {\bf  R. Buckdahn:} Peng's stochastic maximum principle for mean-field stochastic control problems. 
\item 16.50 - 17.25 {\bf  F. Camilli:} Analysis and approximation of   stationary MFG systems  on  networks.
\end{itemize}
% \vskip -2.0cm
% \begin{displaymath}
%  \!\!\!\!\!\!\!\!\!\!\!\! \includegraphics[width=3.2cm]{IHP2.pdf} \!\!\!\!\!\!\!\!\!\!\!\!   \includegraphics[width=3.2cm]{logo_chaire.pdf}  \!\!\!\!\!\!\!\!\!\!\!\! \includegraphics[width=3.2cm]{Logo_ILB_Labex_300x150.pdf} \!\!\!\!\!\!\!\!\!\!\!\! \includegraphics[width=3.2cm]{logoLJLL.pdf}\!\!\!\!\!\!\!\!\!\!\!\! \includegraphics[width=3.2cm]{logo_ceremade3.pdf}\!\!\!\!\!\!\!\!\!\!\!\! \includegraphics[width=3.2cm]{logoUFI-UIF.pdf}
% \end{displaymath}
\vfill\eject
\subsection*{Thursday June  11: Amphith{\'e}{\^a}tre Darboux, IHP}
\begin{itemize}
\item 9.30 - 10.05 {\bf J.-M. Lasry:} Economics of mining industries : MFG approach.\item
10.05 - 10.40  {\bf  G. Turinici:} Vaccination as MFG equilibriums.\item
10.40 - 11.10   coffee break\item
11.10 - 11.45 {\bf  P.A. Lehalle:}  TBA.\item
11.45 - 12.20 {\bf  B. Moll:}  PDE models in Macroeconomics. \item
12.20 - 13.50  Lunch break\item
13.50 -14.25 {\bf  P. Degond:} From kinetic to macroscopic models through local Nash equilibria. \item
14.25 - 15.00 {\bf M. Huang:} Mean Field Growth Modeling with Cobb-Douglas Production and Relative Consumption.\item
15.30 - 16.05  {\bf  V. Kolokoltsov:} On the analysis of the Mean field games with comomn noise via particle systems with generalized binary interaction. \item
16.05 - 16.35  coffee break \item
16.35 - 17.10 {\bf   D. Gomes:} Existence of weak solutions to MFG through monotonicity methods.\item
17.10 - 17.45 {\bf   P.-N. Giraud:} Long term dynamic of mining industries. 
\end{itemize}

 \vfill
%\vskip -2.0cm
% \begin{displaymath}
%  \!\!\!\!\!\!\!\!\!\!\!\! \includegraphics[width=3.2cm]{IHP2.pdf} \!\!\!\!\!\!\!\!\!\!\!\!   \includegraphics[width=3.2cm]{logo_chaire.pdf}  \!\!\!\!\!\!\!\!\!\!\!\! \includegraphics[width=3.2cm]{Logo_ILB_Labex_300x150.pdf} \!\!\!\!\!\!\!\!\!\!\!\! \includegraphics[width=3.2cm]{logoLJLL.pdf}\!\!\!\!\!\!\!\!\!\!\!\! \includegraphics[width=3.2cm]{logo_ceremade3.pdf}\!\!\!\!\!\!\!\!\!\!\!\! \includegraphics[width=3.2cm]{logoUFI-UIF.pdf}
% \end{displaymath}
\eject
\subsection*{Friday June  12: Amphith{\'e}{\^a}tre Chaudron, ENSPC Paris-Tech}
\begin{itemize}
\item 9.30 - 10.05 {\bf  A. Bensoussan:} On the Interpretation of the Master equation.\item
10.05 - 10.40  {\bf J. Frehse:}  TBA. \item
10.40 - 11.10   coffee break \item
11.10 - 11.45 {\bf  W. Gangbo:}  Existence of a solution to an equation arising from Mean Field Games.\item
11.45 - 12.20 {\bf  F. Santambrogio:}  p for price, or p for pressure? analysis of a MFG model under density constraint.\item
12.20 - 13.50  Lunch break\item
13.50 -14.25  {\bf J.D. Benamou:} Augmented Lagrangian numerical methods for variational Mean-Field Games.\item
14.25 - 15.00 {\bf  P. Caines:} Partially Observed Mean Field Games with a Major Player.\item
15.35   Closure

\end{itemize}

\vfill
%  \vskip -2.0cm
% \begin{displaymath}
%  \!\!\!\!\!\!\!\!\!\!\!\! \includegraphics[width=3.2cm]{IHP2.pdf} \!\!\!\!\!\!\!\!\!\!\!\!   \includegraphics[width=3.2cm]{logo_chaire.pdf}  \!\!\!\!\!\!\!\!\!\!\!\! \includegraphics[width=3.2cm]{Logo_ILB_Labex_300x150.pdf} \!\!\!\!\!\!\!\!\!\!\!\! \includegraphics[width=3.2cm]{logoLJLL.pdf}\!\!\!\!\!\!\!\!\!\!\!\! \includegraphics[width=3.2cm]{logo_ceremade3.pdf}\!\!\!\!\!\!\!\!\!\!\!\! \includegraphics[width=3.2cm]{logoUFI-UIF.pdf}
% \end{displaymath}
\eject

\section*{Abstracts}
\paragraph{ M. Bardi {\sl (Universit{\`a}{\^A}� di Padova)}:  Mean field games models of segregation}
The Nobel Laureate Thomas Schelling studied the behaviour of 
different ethnic groups in american cities in the 60s and used simple 
simulations to show that a mild preference of each person for not 
belonging to a too small minority in his/her neighborhood leads to a 
population distribution where each neighborhood is inhabited mostly by a
 single group: the segregation phenomenon. We propose some mean field 
games models of two populations each minimizing a functional that 
describes the preferences observed by Schelling. We show some existence 
results for the stationary as well as the evolutive system of PDEs with 
Neumann boundary conditions. Uniqueness is not expected in these models,
 as it is easy to exhibit examples with multiple equilibria. We present 
numerical simulations obtained by various methods.  They show some form 
of segregation if the noise affecting the dynamics is small, but also 
more complicated and unstable behavior for intermediate noise 
intensities.
This is joint work with Marco Cirant (Universit{\`a}{\^A}� di Milano) and Yves 
Achdou (Universit{\'e} Paris-Diderot).

\paragraph{ A. Bensoussan   {\sl  (The University of Texas   and  City University of Hong Kong)}:
    The interpretation of the master equation.}
 Since its introduction by P.L. Lions in his lectures and seminars at the
    College de France, see also the very helpful notes of Cardialaguet on
    Lions' lectures, the Master equation has attracted a lot of interest, and
    various points of view have been expressed, see Carmona-Delarue,
    Bensoussan-Frehse-Yam , Buckdahn-Li-Peng-Rainer . There are several ways
    to introduce this type of equation. It involves an argument which is a
    probability measure, and P.L. Lions has proposed the idea of working with
    the Hilbert space of random variables which are square integrable. So
    writing the equation is an issue. Another issue is its origin. We discuss
    in this paper these various aspects, and for the modeling rely heavily on
    a seminar at Coll{\`e}ge de France delivered by P.L. Lions on November 14,
    2014.

    \paragraph{ J.-D. Benamou {\sl(INRIA)} : Augmented Lagrangian numerical methods for variational Mean-Field Games}

Hello world!

Many problems from mass transport can be reformulated as 
variational problems under a prescribed divergence constraint (static 
problems) or subject to a time dependent continuity equation which again
 can also be formulated as a divergence constraint but in time and 
space. The variational class of Mean-Field Games introduced by Lasry and
 Lions may also be interpreted as a generalisation of the time-dependent
 optimal transport problem. Following Benamou and Brenier, we show that 
augmented Lagrangian methods are well-suited to treat convex but 
nonsmooth problems and apply to variational Mean Field Games.

\paragraph{R. Buckdahn {\sl(Universit{\'e}  de Brest)}:
  Peng's stochastic maximum principle for mean-field stochastic control problems }
The talk extends Peng's stochastic maximum principle from classical
 stochastic control problems to those in which the coefficients of the 
dynamics of the controlled state process do not only depend on the state
 process and the control themselves but also on the law of the control 
state process. The characterisation of the optimal control, which is 
obtained, extends the corresponding result by Shige Peng and contains 
also extra-terms coming from the mean-field character of the stochastic 
control problem.\\
Joint work with Juan Li (SDU, Weihai), Jin Ma (USC, Los Angeles).

\paragraph{P. Caines     {\sl (McGill University, Montreal)}:
   Partially Observed Mean Field Games with a Major  Player}
Dynamic games with a large population of minor agents and one major agent
are considered where the minor agents partially observe the state of the
major agent. It is shown that the epsilon-Nash equilibrium property holds
in this MFG scenario with the best response control actions of each minor
agent depending upon the conditional density generated by a non-linear
filter for the major agent's state.\\
Work with Nevroz Sen.

\paragraph{F. Camilli {\sl (Sapienza Universit{\`a}{\^A}�  di Roma)}:  Analysis and approximation of   stationary MFG systems  on  networks}
 We study a stationary Mean Field Game system defined on a network.
We  motivate the transition  conditions we consider at the vertices and  we prove 
existence and uniqueness of  a smooth solution to
the system. We also consider  a  numerical scheme  for  the problem: in
  this framework a correct approximation of the transition conditions at the vertices plays a crucial role.
We prove existence, uniqueness and convergence of the scheme and we show some numerical experiments\\
(joint work with S.Cacace (Roma) and C.Marchi (Padova)).

\paragraph{R. Carmona {\sl(Princeton)}: Models of Bank Runs as Mean Field Games of Timing}
We recast several models of bank runs which appeared in the finance literature as Mean Field Games of Timing.
We propose a mathematical framework for these games, and present some existence and approximation results.

 
\paragraph{ P. Degond  {\sl (Imperial College, London)}
  From kinetic to macroscopic models through local Nash equilibria}
 We propose a mean field kinetic model for systems of rational agents
interacting in a game theoretical
framework. This model is inspired from non-cooperative anonymous games with
a continuum of players and Mean-Field Games. The large time behavior of the
system is given by a macroscopic closure with a Nash equilibrium serving as
the local thermodynamic equilibrium. Applications of the presented theory to
social and economical models will be given.

\paragraph{F. Delarue  {\sl (Universit{\'e}  de Nice Sophia-Antipolis)}
   The master equation and the convergence problem in mean-field games}
    We here discuss the construction of a classical solution to the 
master equation associated with a mean-field game, both without and with
 a common noise, under the assumption that the Larry Lions monotonicity 
condition holds true. Then we investigate the convergence of the Nash 
equilibria of the N-player-game to the solution of the mean-field game. 
The key point is to let the solution of the master equation act onto the
 empirical distribution of the system formed by the N players in 
equilibrium: this provides an approximated solution to the N Nash 
system. Taking benefit of the regularity of the solution of the master 
equation, we manage to estimate the distance with the true solution of 
the Nash system.\\
 Based on two joints works with J.F. Chassagneux and D. 
Crisan; and P. Cardaliaguet, J.M. Lasry and P.L. Lions. 

\paragraph{ M. Fischer {\sl (Universit{\`a}{\^A}� di Padova)}:
   On the connection between mean field games and N-player games}
Mean field games arise as limit models for symmetric N-player games 
with interaction of mean field type when the number of players N tends 
to infinity. The limit relation is often understood in the sense that a 
solution of a mean field game allows to construct approximate Nash 
equilibria for the corresponding N-player games. The opposite direction 
is of interest, too: When do sequences of Nash equilibria converge to 
solutions of an associated mean field game? I will discuss recent 
rigorous results in this direction for finite-horizon systems in the 
framework of stochastic open-loop controls. 


\paragraph{ W. Gangbo {\sl(Georgia Institute of Technology)}:  Existence of a solution to an equation arising from Mean Field Games}
 We construct a small time strong solution to a nonlocal 
Hamilton-Jacobi equation
introduced by Lions, the so-called master equation, originating from the
 theory of Mean Field
Games. We discover a link between metric viscosity solutions to local 
Hamilton-Jacobi equations
studied independently by Ambrosio-Feng and Gangbo-Swiech, and the master 
equation. As a consequence
we recover the existence of solutions to the First Order Mean Field 
Games equations, first proved by Lions. We make a more rigorous 
connection between the master equation and the Mean Field
Games equations. \\
(This talk is based on a joint work with A. Swiech).

\paragraph{ P. N. Giraud  {\sl (Cerna, MINES ParisTech)}:
   Long term dynamic of mining industries  }

\paragraph{ D. Gomes {\sl (Universidade Tecnica de Lisboa and  K.A.U.S.T. Saudi Arabia)}:
Existence of weak solutions to MFG through monotonicity methods}
In the present talk, we discuss monotonicity methods for mean-field
 games. We suggest a new definition of weak solution, whose existence 
can be proven under general assumptions. Then, we discuss various 
properties of these weak solutions. Finally, we present applications to 
the numerical approximation of mean-field games.  

\paragraph{  M. Huang {\sl (Carleton University, Ottawa) }:
 Mean Field Growth Modeling with Cobb-Douglas Production and Relative Consumption  }
We consider continuous time mean field consumption-accumulation 
games. The capital stock of each agent evolves according to the 
Cobb-Douglas production function subject to consumption and stochastic 
depreciation. The individual HARA-type utility depends on both the own 
consumption and relative consumption. Under some standard model 
parameters, we analyze the fixed point problem of the mean field game 
and specify it by use of a set of ordinary differential equations. The 
individual strategy is obtained as a linear feedback with the gain 
reflecting the collective behavior of the population. \\
(Joint work with 
Son Luu Nguyen of University of Puerto Rico)

\paragraph{ V. Kolokolstov {\sl(University of Warwick)}:
 On the analysis of the Mean field games with common noise via particle systems with generalized binary interaction }
Games with common noise are attentively studied now by many 
authors.We show under certain conditions that a solution
to the forward-backward limiting system of the mean-field games with 
common noise provides an $1/N$-Nash equilibrium for
the approximating games with $N$ players. Two additional technical ideas
 are used as compared to the standard case without noise: interpretation
 of the common noise as a certain generalized binary interaction and the
 Kunita theory of stochastic characteristics.

\paragraph{ J.M. Lasry  {\sl (Universit{\'e} Paris-Dauphine)}:
   Economics of mining industries : MFG approach }

\paragraph{P.A. Lehalle {\sl(Imperial College, London )}: TBA}


\paragraph{ P-L. Lions  {\sl (Coll{\`e}ge de France)}:
   TBA }

\paragraph{ R. Malham{\'e}  {\sl (Gerad and {\'E}cole Polytechnique de Montr{\'e}al)}:
   Cooperative and Non Cooperative Mean Field Game Methods in Energy Systems }

The high levels of variability and unreliability of renewable energy sources such as wind and
solar energy, act as one of the primary obstacles to their massive adoption for generation in
power systems. In this context, availability of energy storage can help mitigate the increasing
mismatches between load and generation that would result from their generalized use, and
would help limit reliance on more controllable but environmentally damaging fossil based
energy sources for such purpose.
We consider the potential organization of the dispersed energy storage naturally present in
power systems such as found in the thermal inertia of electrically heated or cooled residential
and commercial buildings, but also in electric water heaters and refrigerators for example, into a
giant {\^a}��leaky battery{\^a}�� resource. The challenges are multiple, not the least of which is the
presence of potentially millions of heterogeneous control points where monitoring and
actuation would be required in a classical centralized view of the control problem. Instead we
show how a linear quadratic Mean Field Game formalism provides a natural tool for
decentralization of the controls. The theory is applied to a diffusion model of heating and
cooling loads, and both non cooperative and social solutions are presented with a mix of
theoretical and numerical results. This is joint work with Arman Kizilkale.

\paragraph{B. Moll {\sl(Princeton)}   PDE models in Macroeconomics }
The purpose  is to get mathematicians
interested in studying a number of partial
differential equations (PDEs) that naturally arise
in macroeconomics. These PDEs come from models
designed to study some of the most important
questions in economics. At the same time, they are
highly interesting for mathematicians because their
structure is often quite difficult. We present a number
of examples of such PDEs, discuss what is known
about their properties, and list some open questions
for future research.

\paragraph{ A. Porretta {\sl (Universit{\`a} di Roma Tor Vergata)}: On the weak theory for mean field games systems with local coupling }
The regularity of solutions to mean field games systems can hardly be established for general local 
couplings, depending on the pointwise value of the distribution density. In this case,  it is often 
necessary to work with weak solutions. In this talk I will discuss the well-posedness of those systems in 
the weak setting, explaining how this is related with new results on Fokker-Planck equations through a 
characterization of weak and renormalized solutions. Applications concern several issues, like the 
convergence of numerical schemes or the characterization of solutions to the planning problem.  I will 
also discuss the vanishing viscosity limit and the uniqueness for the relaxed weak formulation in the first 
order case.

\paragraph{ F. Santambroggio {\sl (Universit{\'e}  Paris-Sud)}:
   p for price, or p for pressure? analysis of a MFG model under density constraint}   
  The question of how to replace density penalizations with density 
constraints of the form $\rho\leq 1$ in MFG is a tricky one, in 
particular if one wants a meaningful notion of equilibrium. In the first
 MFG meeting in Rome some years ago, I proposed a model, where the 
movement was affected by the gradient of a pressure, endogenously 
created by the action of the other player, where they saturate the 
density constraint. This model is only formal, and no successful study 
has been possible so far. Moreover, we suspect it to be non-variational.
 In a joint work with P. Cardaliaguet (Paris-Dauphine) and A. Meszaros 
(Paris-Sud), we study a different model, which is the one that we obtain
 as a limit $m\to\infty$ when we add a penalization with the $m$-th 
power of the density. It gives rise to a Benamou-Brenier-type 
optimization problem subject to the constraint $\rho\leq 1$, and admits a
 dual problem where a pressure $p$ appears. The pressure also plays a 
role in the equilibrium, and $p(t,x)$ is, at least formally, the price 
that agents have to pay to pass through $x$ at time $t$. It is 
non-negative, and vanishes where the constraint is not saturated. The 
main difficulty is that $p$ is a priori only a measure, and it does not 
make sense to integrate it along a trajectory. A precise treatise 
requires regularity results very weak but sufficient to provide a 
meaningful and rigorous notion of trajectorial cost: borrowing 
techniques developed by Y. Brenier and then L. Ambrosio and A. Figalli 
for incompressible Euler equations, we can prove that $p$ is $L^2_{loc}$
 in time, valued in $BV$ in space. 

\paragraph{ G. Turinici {\sl (Universit{\'e} Paris-Dauphine)}:
   Vaccination as MFG equilibriums  }


\section*{Sponsors}
\begin{itemize}
\item Institut Henri Poincar{\'e}
\item Chaire Finance et D{\'e}veloppement Durable
\item Institut Louis Bachelier
\item Universit{\'e} Franco-Italienne
\item ANR ISOTACE
\item GDR 2932 Th{\'e}orie des Jeux
\item Ceremade
\item Laboratoire J-L. Lions
\end{itemize}
\end{document}


